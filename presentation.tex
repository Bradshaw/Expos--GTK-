\documentclass{beamer}
\usepackage[french]{babel}
\usepackage[utf8]{inputenc}
\usepackage[T1]{fontenc}
\usepackage{listings}
\usepackage{graphicx}
\DeclareGraphicsExtensions{.pdf,.png,.jpg}
\usetheme{Warsaw}
\title[GTK+]{GTK+\\Une boîte à outils pour interfaces graphiques}
\author{Flavien Godfroy et Kevin Bradshaw}
\date{18 Décembre 2012}
%%configuration de listings
\lstset{
language=c,
basicstyle=\ttfamily\small, %
identifierstyle=\color{black}, %
keywordstyle=\color{blue}, %
stringstyle=\color{pink!200}, %
commentstyle=\it\color{green!95}, %
columns=flexible, %
tabsize=2, %
extendedchars=true, %
showspaces=false, %
showstringspaces=false, %
numbers=left, %
numberstyle=\tiny, %
breaklines=true, %
breakautoindent=true, %
captionpos=b
}
\begin{document}

  \begin{frame}
    \titlepage
  \end{frame}

  \begin{frame}<beamer>
      \frametitle{}
      \tableofcontents
    \end{frame}

  \AtBeginSection[]
  {
    \begin{frame}<beamer>
      \frametitle{}
      \tableofcontents[currentsection]
    \end{frame}
  }

  \section{Introduction}

  \subsection{Qu'est-ce?}
    \begin{frame}{Qu'est-ce?}
    GTK+ est une suite de bibliothèques (toolkit) visant la création d'interfaces graphiques de type WIMP (Windows, Icons, Menu, Pointer) par Widgets.\\
    Visant originalement le langage C, elle est disponible aussi pour beaucoup d'autres, comme C++, Java et Python.
  \end{frame}


  \section{Histoire}
  \subsection{La situation à l'époque}
  \begin{frame}{X Window System}
    \begin{itemize}
      \item Protocole logiciel et réseau pour les interfaces graphiques\\
      \item Créé au MIT en 1984\\
      \item Devenu le standard "de facto" des affichages\\
      \item Ne fournit pas de "look and feel" standardisé
    \end{itemize}
  \end{frame}
  \begin{frame}{OPEN LOOK}
    \begin{figure}[htb]
    \centering
    \includegraphics[scale=0.5]{"openlook"}
    \caption{Un gestionnaire de fichiers avec OPEN LOOK}
    \label{fig:openlook}
    \end{figure}
  \end{frame}
  \begin{frame}{OPEN LOOK}
    \begin{itemize}
      \item Spécification logicielle pour les interfaces graphiques\\
      \item Créé à la fin des années 1980 par Sun Microsystems et AT\&T pour Unix\\
      \item Cherche à définir un "look and feel" pour Unix, pour suivre les efforts de Macintosh, Microsoft et Amiga
    \end{itemize}
  \end{frame}
  \begin{frame}{Motif}
    \begin{figure}[htb]
    \centering
    \includegraphics[scale=0.275]{"motif"}
    \caption{Un calendrier avec Motif}
    \label{fig:motif}
    \end{figure}
  \end{frame}
  \begin{frame}{Motif}
  \begin{itemize}
    \item Un concurrent à OPEN LOOK\\
    \item Finit par remplacer complètement OPEN LOOK en Juin 1993
  \end{itemize}
  \end{frame}

  \subsection{GIMP}
  \begin{frame}{GIMP}
    \begin{figure}[htb]
    \centering
    \includegraphics[scale=0.225]{"gimp"}
    \caption{GIMP rendu avec GTK+}
    \label{fig:gimpgtk}
    \end{figure}
  \end{frame}
  \begin{frame}{GIMP}
  \begin{itemize}
    \item General Image Manipulation Program puis GNU Image Manipulation Program\\
    \item Peter Mattis et Spencer Kimball commencent le projet GIMP à U.C. Berkeley en 1995\\
    \item Mattis n'étant pas satisfait par Motif, il crée le GIMP ToolKit ("GTK")\\
    \item 1997, GIMP 0.60 utilise exclusivement GTK\\
    \item 1998, Tor Lilqvist commence un portage de GTK et GIMP sur Windows\\
    \item GTK se voit reconçu en orienté objet sous le nom "GTK+"
  \end{itemize}
  \end{frame}

  \subsection{Aujourd'hui}
  \begin{frame}{Aujourd'hui}
  \begin{itemize}
    \item GIMP et GTK+ séduisent la communauté GNU\\
    \item Nouveau standard pour beaucoup de logiciels libres\\(ex: GNOME, Firefox, Ardour et Inkscape)\\
    \item GTK+ se lance sur Macintosh et Windows, et encore d'autres\\
    \item Liaisons avec d'autres langages, GTK+ devient plus accessible
    \end{itemize}
  \end{frame}

  \section{GTK+ son fonctionnement}
	\subsection{Le layout manager}
	\begin{frame}{Le layout manager}
		C'est un objet qui va nous permettre de mettre en forme notre fenêtre\\
		Voici différent layout manager :\\
		\begin{itemize}
			\item GtkAlignment
			\item GtkFixed
			\item GtkVBox
			\item GtkTable
		\end{itemize}
	\end{frame}
	\subsection{Les Widget}
	\begin{frame}{Un Widget c'est quoi ?}
		\textbf{Définition : }\\
		En informatique, le mot widget recouvre deux notions distinctes en relation avec les interfaces graphiques , il peut désigner :\\
		\begin{itemize}
			\item Un élément de base d'une interface graphique
			\item Un petit outil qui permet d'obtenir des informations
		\end{itemize}
		Pour GTK tout les objets sont traité comme des widgets\\
	\end{frame}
	\subsection{Première fenêtre}
	\begin{frame}[containsverbatim]{Première fenêtre}
		Nous allons voir comment faire une fenêtre simple comme celle la :\\
		\centering
		\includegraphics[scale=0.35]{"hello"}\\
		\begin{lstlisting}
			//avant de commencer on oublie pas
			#include <gtk/gtk.h>
			//on definit un widget qui sera notre fenetre
				GtkWidget *window;
			//on laisse gtk se charger des configuration par defaut
				gtk_init(&argc, &argv);
		\end{lstlisting}
	\end{frame}
	\begin{frame}[containsverbatim]{Première fenêtre}
		\begin{lstlisting}
			//on creer notre fenetre
				window = gtk_window_new(GTK_WINDOW_TOPLEVEL);
			//on lui donne un titre et une taille
				gtk_window_set_title(GTK_WINDOW(window), "Hello, world!");
				gtk_window_set_default_size(GTK_WINDOW(window), 255, 200);
			//on rend visible notre fenetre
				gtk_widget_show_all(window);
			//on lance le main de gtk qui va attendre une action
				gtk_main();
		\end{lstlisting}
	\end{frame}

	\section{Un petit projet avec GTK+}
	\subsection{Calculatrice}
	\begin{frame}[containsverbatim]{Interface de la calculatrice}
		\centering
		\includegraphics[scale=0.4]{"calc"}\\
		\begin{lstlisting}
			//on creer notre layout manager qui va contenir les boutons
				table = gtk_table_new(4, 4, 1);
				gtk_container_add(GTK_CONTAINER(window), table);
				gtk_container_set_border_width(GTK_CONTAINER(table),5);
		\end{lstlisting}
	\end{frame}
	\begin{frame}[containsverbatim]{Les évènements}
		A chaque widget on peut associer un ou plusieurs évènements avec\\
		\begin{lstlisting}
			g_signal_connect(instance, detailed_signal, c_handler, data)
		\end{lstlisting}
		instance : notre widget sur lequel on souhaite faire le connect\\
		detailed signal : type d'évènement\\
		c handler : la fonction qui sera appelé quand on aura l'évènement\\
		data : les données transmisent à la fonction\\
	\end{frame}
	\begin{frame}[containsverbatim]{Les évènements}
		\begin{lstlisting}
			//Fonction de remise a 0
			void clear(GtkWidget *pBtn, MyData* data){
				gtk_entry_set_text(GTK_ENTRY(data->Entrer), "0");
				data->res = 0;
				data->type = 0;
			}
			//Liaison de la fonction avec le bouton
				g_signal_connect(G_OBJECT(Bouton),  "clicked", G_CALLBACK(clear), &data);
		\end{lstlisting}
	\end{frame}

 \section{Pourquoi GTK+?}
  \subsection{Une librairie multi-plateformes, multi-langages}
  \begin{frame}{Pourquoi GTK+?}
    \begin{figure}[htb]
    \centering
    \includegraphics[scale=0.6]{"gtkplus"}
    \caption{Un gestionnaire de fichiers avec GTK+}
    \label{fig:gtk}
    \end{figure}
  \end{frame}
  \begin{frame}{Une librairie multi-plateformes, multi-langages}
  \begin{itemize}
    \item Offre une apparence unifiée sur tous les systèmes principaux\\
    \item Disponible pour la majorité des langages utilisés en entreprise\\
    \item Facile à lier à un nouveau langage si besoin est\\
    \item Rend le portage de projets sur d'autres systèmes très simple\\
    \item A su prouver sa fiabilité dans les différents projets qui l'utilisent
  \end{itemize}
  \end{frame}
  \subsection{Comparaison à Qt}
  \begin{frame}{Qt}
    \begin{figure}[htb]
    \centering
    \includegraphics[scale=0.45]{"qt"}
    \caption{Un editeur de contacts avec Qt}
    \label{fig:qt}
    \end{figure}
  \end{frame}
  \begin{frame}{Comparaison à Qt}
  \begin{itemize}
    \item Le concurrent pricipal\\
    \item Qt fournit aussi des outils pour le multi-processus, multimedia, réseau et d'autres\\
    \item Qt est implémenté directement en C++, GTK+ en C\\
    \item Les deux sont compatibles avec la majorité des langages et systèmes d'exploitation principaux
  \end{itemize}
  \end{frame}

  \section{Sources}
  \begin{frame}{Sources}
  	\url{http://zetcode.com/tutorials/gtktutorial}\\
    \url{http://aruiz.typepad.com/siliconisland/2008/03/gtk-hackfest-20.html}\\
		\url{http://web.archive.org/web/19990417052141/http://www.linuxworld.com/linuxworld/lw-1999-01/lw-01-gimp.html}\\
		\url{http://developer.gnome.org/gtk3/stable/}\\
    \url{http://www.gtk.org/language-bindings.php}\\
    \url{http://developer.gnome.org/gtk-faq/stable/x90.html}\\
  \end{frame}
\end{document}
