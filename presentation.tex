\documentclass{beamer}
\usepackage[french]{babel}
\usepackage[utf8]{inputenc}
\usepackage[T1]{fontenc}
\usetheme{Warsaw}
\title[GTK+]{GTK+\\Une boîte à outils pour interfaces graphiques}
\author{Flavien Godefroy et Kevin Bradshaw}
\date{18 Décembre 2012}
\begin{document}

  \begin{frame}
    \titlepage
  \end{frame}

  \begin{frame}<beamer>
      \frametitle{}
      \tableofcontents
    \end{frame}

  \AtBeginSection[]
  {
    \begin{frame}<beamer>
      \frametitle{}
      \tableofcontents[currentsection]
    \end{frame}
  }

  \section{Introduction}

  \subsection{Qu'est-ce?}
    \begin{frame}{Qu'est-ce?}
    GTK+ est une suite de bibliothèques (toolkit) visant la création d'interfaces graphiques de type WIMP (Windows, Icons, Menu, Pointer) par Widgets.\\
    Visant originalement le langage C, elle est disponible aussi pour beaucoup d'autres, comme C++, Java et Python.
  \end{frame}


  \section{Histoire}
  \subsection{La situation à l'époque}
  \begin{frame}{X Window System}
    Protocole logiciel et réseau pour les interfaces graphiques\\
    Créé au MIT en 1984\\
    Devenu le standard "de facto" des affichages\\
    Ne fournit pas de "look and feel" standardisé
  \end{frame}
  \begin{frame}{OPEN LOOK}
    Spécification logicielle pour les interface graphiques\\
    Créé à la fin des années 1980 par Sun Microsystems et AT\&T pour Unix\\
    Cherche à définir un "look and feel" pour Unix, pour suivre les efforts de Macintosh, Microsoft et Amiga
  \end{frame}
  \begin{frame}{Motif}
    Un compétiteur à OPEN LOOK\\
    Finit par remplacer complètement OPEN LOOK en Juin 1993
  \end{frame}

  \subsection{GIMP}
  \begin{frame}{GIMP}
    General Image Manipulation Program puis GNU Image Manipulation Program\\
    Peter Mattis et Spencer Kimball commencent le projet GIMP à U.C. Berkeley en 1995\\
    Mattis n'était pas satisfait par Motif, il crée GTK, pour GIMP ToolKit\\
    Remplace totalement Motif en 1997 avec la version 0.60 de GIMP\\
    1998, Tor Lilqvist commence le portage de GTK et GIMP sur Windows, et GTK est reconçu en orienté objet et se voit renommé GTK+
  \end{frame}

  \subsection{GTK\#, gtkmm, PyGTK, etc...}
  \begin{frame}{GTK\#, gtkmm, PyGTK, etc...}
    GIMP et GTK+ séduisent la communauté GNU\\
    Nouveau standard pour beaucoup de logiciels libres comme GNOME, Firefox, Ardour et Inkscape\\
    GTK+ devient compatible avec Macintosh et Windows, et même récemment HTML5\\
    Des liaisons sont crées pour d'autres langages, rendant GTK+ accéssible à la majorité des développeurs.
  \end{frame}


  \section{Comparaison à ses competiteurs aujourd'hui}
  \subsection{Qt}
  \begin{frame}{Qt}
    Le concurrent pricipal\\
    Qt fournit aussi des outils pour le multi-processus, multimedia, réseau et d'autres\\

  \end{frame}

  \section{Un petit projet avec GTK+}
  \subsection{Calculatrice}
  \begin{frame}{Finger in the nose!}
    La classe ultime
  \end{frame}


  \section{Pourquoi GTK+?}
  \subsection{Bonne question}
  \begin{frame}{La réponse?}
    Je n'en sais guère...
  \end{frame}

\end{document}